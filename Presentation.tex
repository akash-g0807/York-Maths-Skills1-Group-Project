%----------------
% Math Skills 1: Beamer template
%-----------------

\documentclass[10pt]{beamer}

\usetheme{Madrid}

\mode<presentation>

\title[Numerical Methods]{Numerical Method for ODE's}

\author[Akash Gopinath]{Akash Gopinath}
\institute[Maths, York]{University of York, UK }
%%%%%%%%% PREAMBLE %%%%%%%%%%%


\usepackage{graphicx}
\usepackage{graphicx, epstopdf}
\usepackage{caption}
\usepackage{subcaption}

\usepackage{lmodern}
\usepackage{tcolorbox}

\newtcolorbox{mybox}[3][]
{
	colframe = #2!100,
	colback  = #2!10,
	coltitle = #2!20!black,  
	#1,
}



\usepackage{color} %red, green, blue, yellow, cyan, magenta, black, white
\definecolor{mygreen}{RGB}{28,172,0} % color values Red, Green, Blue
\definecolor{mylilas}{RGB}{170,55,241}
\definecolor{myblue}{RGB}{51,51,178}

\usepackage{amsmath}
\usepackage{amsrefs}
\usepackage{amsmath}
\newtheorem{defn}{Definition}
\newtheorem{propn}{Proposition}
\newtheorem{thm}{Theorem}
\newtheorem{cor}{Corollary}
\usepackage{xpatch}

\xpatchcmd{\proof}
{\itshape}
{\bfseries}
{}
{}



\newcommand{\ndiv}{\hspace{-4pt}\not|\hspace{2pt}}
%%%%%%%%% PREAMBLE %%%%%%%%%%%%
\begin{document}
	
	\maketitle
	
	\begin{frame}
		\frametitle{Why use Numerical Methods}
		\begin{itemize}
			\item So far in our course we have been solving Ordinary Differential Equations \textbf{analytically} i.e. we have been finding {\bf exact} solutions.
			
			\vspace{5px}
			
			\item But a lot of the problems we solve may not have analytical solutions or might be too difficult to solve analytically.
			
			\vspace{5px}
			
			\item And a lot of the time, the problems we are trying to solve will be too complicated and computers will be employed to help us solve the problem.
			
			\vspace{5px}
			
			\item In both of these an many other cases, Numerical Methods are employed
			
			\vspace{5px}
			
			\item {\bf Note:} Numerical methods do not give exact {\bf solutions} but {\bf approximations}
		\end{itemize}
	\end{frame}


	\begin{frame}
		\frametitle{Euler's Method}
		\begin{itemize}
			\item The method we will be discussing today is Euler's Method
				\begin{mybox}{myblue}{A blue box}
			\begin{center}
				\underline{\textbf{Euler's Method Algorithm:}}
			\end{center}
			Let:
			$$\frac{dy}{dt} = f(y,t)$$
			be an ODE with solution $y(t)$ on the interval $[a,b]$ with initial value $y(a) = y(t_0) = y_0$\\
			Let $\displaystyle t_k = t_{k-1} + h  \Rightarrow t_k = t_0 + kh$ where $$\displaystyle h = \frac{b-a}{n}$$ and n is the number of data points. Then
			$$\boldsymbol{y_k = y_{k-1} + f(t_{k-1}, y_{k-1})h}$$
			where $y_k \approx y(t_k)$ for $t_k$ where $k \in \{1, \ldots, n\}$.
			\end{mybox}
		\end{itemize}
	\end{frame}

	\begin{frame}
		\begin{itemize}
			\item The key {\bf mathematical idea} is:
			\begin{mybox}{myblue}{A blue box}
			 At each step the approximation of the graph of the unknown solution $y(t)$ is done by the {\bf tangent line}.
			\end{mybox}
		% \pause
		\item In this presentation I will show you the algorithm in play by solving the following initial value problem (IVP):
		$$\frac{dy}{dx} = e^x \ \ \ \ \text{on the interval} \ \ [2,3]$$
		\item {\bf Note:} This example was only chosen for simplicity. I could have chosen many other examples.
		\item The exact solution to this IVP (not surprisingly) is:
		$$y = e^x$$
			
		\end{itemize}
	\end{frame}


	\begin{frame}
		\frametitle{Euler's Method in action!!!!!!!}
		\begin{itemize}
			
			\item First I will show you how it works for just two {\bf 2 data points} 1 iteration at a time:
			
			\begin{enumerate}
				\item Iteration 1:
					\begin{figure}[h]
						\includegraphics[width=0.6\textwidth,height=0.6\textheight,keepaspectratio]{graph1.eps}
					\end{figure}
			\end{enumerate}
		
	\end{itemize}
	\end{frame}

	\begin{frame}
		
		\begin{itemize}
			\item The second iteration:
			
			\begin{enumerate}
				\setcounter{enumi}{1}
				\item Iteration 2:
				\begin{figure}[h]
					\includegraphics[width=0.6\textwidth,height=0.6\textheight,keepaspectratio]{graph2.eps}
				\end{figure}
			\end{enumerate}
			
		\end{itemize}
		
	\end{frame}

	\begin{frame}
		\frametitle{Wait a minute!, that is not a good approximation...}
		Well as you can see the approximation does not look very good. But lets see what happens when we increase the number of data points
		\begin{itemize}
			\item When n = 3, i.e. We use 3 data points:
		\begin{figure}[h]
			\includegraphics[width=0.6\textwidth,height=0.6\textheight,keepaspectratio]{Graph2,3,3.eps}
		\end{figure}
		\end{itemize}
	\end{frame}

	\begin{frame}
		
		\begin{itemize}
			\item When n = 10, i.e. We use 3 data points:
			\begin{figure}[h]
				\includegraphics[width=0.6\textwidth,height=0.6\textheight,keepaspectratio]{Graph2,10,3.eps}
			\end{figure}
		\end{itemize}
	
	\end{frame}

		\begin{frame}
		
		\begin{itemize}
			\item When n = 100, i.e. We use 3 data points:
			\begin{figure}[h]
				\includegraphics[width=0.6\textwidth,height=0.6\textheight,keepaspectratio]{Graph2,100,3.eps}
			\end{figure}
		\end{itemize}
		
	\end{frame}

\end{document}
